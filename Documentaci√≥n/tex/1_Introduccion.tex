\capitulo{1}{Introducción}

Este proyecto comienza con la necesidad de crear una aplicación web que permita estimar diferencias y composición de la dieta de vertebrados herbívoros. Este problema está definido en el artículo: \emph{''Using n-alkanes to estimate diet composition of herbivores: a novel mathematical approach''} \cite{problemn-alkanes2007} en el cual se plantea la resolución de un sistema de ecuaciones con restricciones, con el objetivo de calcular la composición de la dieta de diferentes especies de herbívoros. Esto problema pretende responder a la pregunta de ¿Cuáles son las proporciones máximas y mínimas de cada componente de la dieta?.

Los modelos de mezcla son utilizados para estimar contribuciones de diferentes fuentes a una mezcla. Estos modelos suelen requerir datos que actúan como marcadores que caractericen rasgos (comúnmente químicos) de las fuentes y las mezclas. Los modelos de mezcla están muy presentes en el ámbito de  la ecología, utilizando isótopos estables como trazadores para evaluar la contribución de los componentes de la dieta (fuentes) a la dieta de los consumidores (mezcla). Aunque este problema tiene otras muchas aplicaciones: movimiento de animales, origen de los contaminantes, transferencia de nutrientes entre ecosistemas, huellas de la erosión de los sedimentos y evaluar las relaciones entre depredadores y presas desde perfiles de ácidos grasos \cite{bicknell:2014,bartov:2012,granek:2009,blake:2012,neubauer:2015}.

Es este trabajo de fin de grado, se ha diseñado una aplicación web que permite resolver sistemas de ecuaciones con restricciones, para lo cual se ha hecho uso de algunas librerías \emph{open source} que han facilitado la resolución del problema (librería GLPK \cite{glpk:package} y algunos paquetes Node). 

\section{Estructura de la memoria}

La estructura de este documente viene formada por siete capítulos:

\begin{itemize}
    \item \textbf{1. Introducción:} Una descripción breve del marco donde se encuentra. Y la estructura de la memoria y los anexos.
    \item \textbf{2. Objetivos del proyecto:} hemos definido los requisitos de este proyecto, distinguiendo entre generales, funcionales y técnicos.
    \item \textbf{3. Conceptos teóricos:} Exposición conocimientos para facilitar la compresión del proyecto.
    \item \textbf{4. Técnicas y herramientas:} Listado de metodologías y herramientas utilizadas en el proyecto
    \item \textbf{5. Aspectos relevantes del desarrollo del proyecto:} Se ha resumido el proceso de desarrollo del proyecto, destacando aspectos claves del mismo.
    \item \textbf{6. Trabajos relacionados:} Se ha realizado una búsqueda de trabajos similares al software presentado.
    \item \textbf{7. Conclusiones y Líneas de trabajo futuras:} Conclusión personal del proyecto y posibles vías de trabajo futuras.
\end{itemize}

\section{Estructura de los anexos}

\begin{itemize}
    \item \textbf{Plan de Proyecto Software:} Planificación temporal y estudio de viabilidad económica y legal.
    \item \textbf{Especificación de Requisitos:} Objetivos y requisitos iniciales del proyecto.
    \item \textbf{Especificación de diseño:} Recoge lo referente al diseño de la aplicación, incluyendo diseño de datos, procedimental y arquitectónico.
    \item \textbf{Documentación técnica de programación:} Explica la estructura de los directorios del repositorio GitHub, junto con manuales orientados al programador (instalación, compilación y ejecución del proyecto).
    \item \textbf{Documentación de usuario:} Es una guía de uso de la aplicación paso por pasos con capturas de la interfaz.
\end{itemize}