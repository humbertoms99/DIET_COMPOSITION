\capitulo{2}{Objetivos del proyecto}

En este apartado se definen los requisitos del proyecto distinguiendo entre objetivos generales, requisitos de software y objetivos técnicos.

\section{Objetivos generales}

El objetivo principal de este proyecto es implementar una aplicación web que resuelva problemas de programación lineal dentro del contexto de \emph{Mixing models} (modelos de mezcla). Esta aplicación deberá ser accesible desde cualquier sistema operativo y móvil. La aplicación web se  implementará como una Single-page application. La vista de la interfaz deberá ser simple e intuitiva.

\section{Objetivos de funcionalidad}

\begin{itemize}
    \item \textbf{Usabilidad y diseño:} La aplicación web debe ser fácil de usar para el usuario, y fácil de entender para los usuarios que estén familiarizados con el problema.
    \item \textbf{Accesibilidad:} La web debe ser accesible por la mayor parte de navegadores en los principales sistemas operativos.
    \item \textbf{Importación y Exportación:} Se debe permitir la descarga de las soluciones y la importación de los datos del problema. %Con el objetivo de aportar una mayor agilidad en el manejo de la aplicación.
    \item \textbf{Internacionalización:} La aplicación debe incorporar todos los textos tanto en Español como en Inglés. Asimismo, deberá permitir la incorporación de idiomas adicionales sin que esto conlleve a realizar cambios en el código.
\end{itemize}

\section{Objetivos técnicos}

\begin{itemize}
    \item Utilización de la metodología Scrum para el seguimiento y el control del proyecto, siendo necesario trabajar con herramientas de gestión de proyectos y de control de versiones durante el desarrollo.
    \item Selección y aprendizaje de un lenguaje de programación para desarrollar una aplicación web .
	\item Creación de una aplicación web que realice  peticiones web a una API .
	\item Generación de ficheros .csv a partir de los  datos resultantes. Exportaciones de los cálculos intermedios, de la solución  junto con los datos del problema y un último documento que tenga la estructura permitida en la carga de datos. La visualización de los informes debe ser similar a visual de  la aplicación web.
	\item Carga de datos por medio de ficheros .csv.
	\item Utilización de librerías para resolver problemas de programación lineal.
	\item La aplicación deberá contener un apartado de guía, que tendrá que ser accesible desde la aplicación en cualquier momento.
	\item Interfaz gráfica simple y limpia.
    \item Utilización de herramientas de análisis de código.
\end{itemize}