\capitulo{7}{Conclusiones y Líneas de trabajo futuras}

En este capítulo incluiré mis conclusiones. Además, se citarán algunas posibles mejoras al proyecto o posibles líneas de desarrollo futuras.

\section{Conclusiones}

El tiempo que he dedicado al trabajo de fin de grado me ha permitido obtener muchas habilidades nuevas, especialmente sobre el desarrollo de aplicaciones web, aprendiendo los fundamentos de los principales lenguajes que se utilizan en estas aplicaciones. Se ha conseguido desarrollar una aplicación Single-page Application en Angular; por otra parte, se han utilizado librerías que resuelven problemas lineales utilizando una librería escrita en Typescript. Se aprendió a realizar peticiones desde una API a una web usando textos en JSON y apoyándome en Postman para comprobar el funcionamiento de las peticiones. También querría destacar el uso de estilos, principalmente por medio los Componentes que ofrece Bootstrap. Y también se ha aprendido sobre \LaTeX{}, que es un lenguaje de creación de documentos que se ha utilizado para realizar la documentación de este proyecto. Por último, he aplicado las metodologías ágiles para la planificación de este proceso.

\section{Posibles mejoras}
Algunas mejoras que consideramos interesantes ante posibles desarrollos futuros son:

\begin{itemize}
    \item \textbf{Mejorar la interfaz gráfica:} En la aplicación se han utilizado principalmente Componentes Bootstrap prediseñados, y se han reajustado algunos colores y posiciones, pero creemos que los estilos pueden aportar más calidad a la aplicación. Algunas nuevas mejoras serian hacer la aplicación más adaptativa (Responsive), añadir Spinner en los tiempos de carga, o aplicar un estilo más similar en toda la aplicación. 
    \item \textbf{App Android:} Creación de una App para móviles, añadiendo mejoras de diseño de interfaz y experiencia de usuario para los usuarios móviles, y diseño de un logo de la aplicación
    \item \textbf{Permita resolver otros problemas:} Añadir la resolución de otros problemas matemáticos que estén relacionados con problemas de Mixing Models.
    \item \textbf{Análisis de datos:} Añadir análisis de los resultados obtenidos, con sus correspondientes botones de exportación.
    %\item \textbf{Incluir manual de usuario: } En la aplicación de añadió un apartado con un vídeo a modo de guía de la aplicación, pero seria interesante añadir un manual de usuario que añada mas detalle sobre el funcionamiento de la aplicación.
\end{itemize}
